% !TeX spellcheck = pt_BR
% Latex template: mahmoud.s.fahmy@students.kasralainy.edu.eg
% For more details: https://www.sharelatex.com/learn/Beamer

\documentclass{beamer}					% Document class

\usepackage[english]{babel}				% Set language
\usepackage[utf8x]{inputenc}			% Set encoding

\mode<presentation>						% Set options
{
  \usetheme{default}					% Set theme
  \usecolortheme{default} 				% Set colors
  \usefonttheme{default}  				% Set font theme
  \setbeamertemplate{caption}[numbered]	% Set caption to be numbered
}

% Uncomment this to have the outline at the beginning of each section highlighted.
%\AtBeginSection[]
%{
%  \begin{frame}{Outline}
%    \tableofcontents[currentsection]
%  \end{frame}
%}

\usepackage{graphicx}					% For including figures
\usepackage{booktabs}					% For table rules
\usepackage{hyperref}					% For cross-referencing

\title{Otimização e algoritmos}	% Presentation title
\author{Bla,Bla, and bla}	% Presentation author
\institute{IST}					% Author affiliation
\date{\today}									% Today's date	

\begin{document}

% Title page
% This page includes the informations defined earlier including title, author/s, affiliation/s and the date
\begin{frame}
  \titlepage
\end{frame}

% Outline
% This page includes the outline (Table of content) of the presentation. All sections and subsections will appear in the outline by default.
\begin{frame}{Outline}
  \tableofcontents
\end{frame}

% The following is the most frequently used slide types in beamer
% The slide structure is as follows:
%
%\begin{frame}{<slide-title>}
%	<content>
%\end{frame}

\section{Part 3}

\begin{frame}{Parte 3  - Introdução}
Nesta parte pretende-se resolver o seguinte problema
\begin{equation*} \label{eq:problem}
\begin{array}[t]{ll} 
\underset{\mathbf{y} \in \mathbf{R}^{Nk}}{\text{minimize}} & f(\mathbf{y}),\\
\end{array} 
\end{equation*}
onde
\begin{equation*}\label{eq:deff}
f(\mathbf{y}) := \sum_{m=1}^{N}\sum_{n=m+1}^{N}\left(||\mathbf{y_m}-\mathbf{y_n}||_2-D_{mn}\right)^2 = \sum_{m=1}^{N}\sum_{n=m+1}^{N} f_{mn}(\mathbf{y})^2 \:,
\end{equation*} 
e
\begin{equation*}\label{key}
D_{mn} = ||\mathbf{x_m}-\mathbf{x_n}||_2 \:.
\end{equation*}
\begin{itemize}
\item Não é um problema de otimização convexo!
\end{itemize}
\end{frame}

\begin{frame}{Part 3  - Task 1}
	O dataset desta \textit{task} foi carregado e calculou-se a matriz $D$ de acordo com 
	\begin{equation*}\label{key}
D_{mn} = ||\mathbf{x_m}-\mathbf{x_n}||_2 \:.
\end{equation*}
obtendo-se
\begin{equation*}\label{key}
D_{2,3} = 5.8749, \quad D_{4,5} = 24.3769
\end{equation*} 
e
\begin{equation*}\label{key}
\mathrm{max}(D_{mn}) = 83.003 \quad \text{for} \quad (m,n) \in \{(134,33),(33,134)\}\:.
\end{equation*}\end{frame}

\begin{frame}{Part 3  - Task 2}
	content...
\end{frame}


\begin{frame}{Part 3  - Task 3}
	content...
\end{frame}


%\begin{frame}{Slide with bullet points}
%	This is a bullet list of two points:
%    \begin{itemize}
%		\item Point one
%        \item Point two
%	\end{itemize}
%\end{frame}
%
%\begin{frame}{Slide with two columns}
%	\begin{columns}
%		\column{.5\textwidth}
%        Text goes in first column.
%        
%        \column{.5\textwidth}
%        Text goes in second column
%	\end{columns}
%\end{frame}
%
%\section{Section Two}
%
%\begin{frame}{Slide with table}
%	\input{tables/table1.tex}
%\end{frame}
%
%\begin{frame}{Slide with figure}
%	\begin{figure}[H]
%		\centering
%        \includegraphics[width=.5\textwidth]{figures/figure1.png}
%        \caption{Caption for figure one.}
%        \label{fig:figure1}
%	\end{figure}
%\end{frame}
%
%\begin{frame}{Slide with references}
%	This is to reference a figure (Figure \ref{fig:figure1})\\
%    This it to reference a table (Table \ref{tab:table1})\\
%    This is to cite an article \cite{Ahmed2018a}\\
%    This is to add an article to the references without mentioning in the text \nocite{Ahmed2018a}\\
%\end{frame}
%\section{References}
%
%% Adding the option 'allowframebreaks' allows the contents of the slide to be expanded in more than one slide.
%\begin{frame}[allowframebreaks]{References}
%	\tiny\bibliography{references}
%	\bibliographystyle{apalike}
%\end{frame}

\end{document}
